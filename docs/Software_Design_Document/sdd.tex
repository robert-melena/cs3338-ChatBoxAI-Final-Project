\documentclass[12pt]{article}

\usepackage[margin=1in]{geometry}
\usepackage{fancyhdr}
\usepackage{hyperref}
\usepackage{longtable}
\usepackage{graphicx}

\pagestyle{fancy}
\fancyhf{}
\rhead{ChatBox AI – SDD}
\lhead{Software Design Document}
\cfoot{\thepage}

\title{ChatBox AI \\ \Large Software Design Document (SDD)}
\author{Robert Melena, Miguel Aguirre, Robert Guiterrez  \\ California State University, Los Angeles}
\date{\today}

\begin{document}
\maketitle
\thispagestyle{empty}
\newpage

\tableofcontents
\newpage

%--------------------------------------------------
\section*{Version History}
\addcontentsline{toc}{section}{Version History}

\begin{longtable}{|p{2.5cm}|p{3cm}|p{6cm}|p{3cm}|}
\hline
\textbf{Version} & \textbf{Date} & \textbf{Description} & \textbf{Author} \\ \hline
1.0 & Snapshot 1 & Initial draft of SDD: architecture, basic UI description & Robert Melena, Miguel Aguirre, Robert Guiterrez \\ \hline
1.1 & Snapshot 2 & Added design for large feature \emph{Conversation History and Export} & Robert Melena, Miguel Aguirre, Robert Guiterrez \\ \hline
1.2 & Snapshot 3 & Added design for Admin Dashboard metrics and data model refinements & Your Name \\ \hline
1.3 & Snapshot 4 & Final polishing, added future work and updated workflow & Robert Melena, Miguel Aguirre, Robert Guiterrez\\ \hline
\end{longtable}

\newpage

%==================================================
\section{Introduction}

\subsection{Purpose}
The purpose of this Software Design Document (SDD) is to describe the overall architecture, components, and detailed design of \textbf{ChatBox AI}, a web-based intelligent chat assistant targeted at university students. This document is intended to guide the hypothetical implementation, testing, and maintenance activities throughout the 12-month project timeline.

\subsection{Intended Audience}
This document is intended for:
\begin{itemize}
  \item The course instructor reviewing the project design.
  \item Team members responsible for implementation, testing, and documentation.
  \item Future student teams who may want to extend or reuse this design.
\end{itemize}

\subsection{Scope and Overview}
ChatBox AI will provide:
\begin{itemize}
  \item A web UI for students to ask questions in natural language.
  \item Backend services to route messages to an AI model.
  \item Persistent storage of conversation history and user profiles.
  \item Basic admin tools to view usage metrics and manage configuration.
\end{itemize}
\noindent
This document explains system architecture, UI structure, database concepts, and integration points with external tools (Docker, TestRail, Jira).

%==================================================
\section{System Architecture}

\subsection{High-Level Architecture Overview}
At a high level, ChatBox AI follows a typical three-tier architecture:
\begin{itemize}
  \item \textbf{Client (Frontend):} Web browser rendering a single-page application or simple web interface.
  \item \textbf{Application Server (Backend):} Exposes REST APIs for chat sessions, user accounts, and admin features.
  \item \textbf{Data Layer:} Relational database for user and chat data; optional vector store for AI context.
\end{itemize}

The system will be containerized using Docker and composed using \texttt{docker-compose}. A reverse proxy (e.g., NGINX) may sit in front of the frontend and backend for routing.

\subsection{Component Diagram (Conceptual)}
Key components include:
\begin{itemize}
  \item \textbf{Web UI} – Chat interface, login/register, settings, admin dashboard.
  \item \textbf{API Gateway / Backend Service} – Handles authentication, routing, rate limiting, and business logic.
  \item \textbf{AI Adapter} – Encapsulates calls to an external Large Language Model (LLM) provider.
  \item \textbf{Database} – Stores users, sessions, messages, and configuration.
  \item \textbf{Logging and Monitoring} – Captures logs, basic metrics (request count, response times).
\end{itemize}

\subsection{Deployment View}
In a Docker-based deployment:
\begin{itemize}
  \item One container runs the \textbf{frontend} (static assets served via NGINX).
  \item One container runs the \textbf{backend service} (e.g., Node.js, Python, or Java Spring Boot).
  \item One container runs the \textbf{database} (e.g., PostgreSQL).
  \item Optionally, one container runs a \textbf{vector database} (e.g., a simple text index or placeholder for a real vector store).
\end{itemize}

The \texttt{docker-compose.yml} file in the repository describes these services, networks, and environment variables at a high level.

%==================================================
\section{Detailed Design}

\subsection{Backend Modules}

\subsubsection{Authentication Module}
\begin{itemize}
  \item Handles sign up, login, and token-based authentication.
  \item Stores password hashes in the database (no plain-text storage).
\end{itemize}

\subsubsection{Chat Session Module}
\begin{itemize}
  \item Creates chat sessions for each user.
  \item Stores each message (user and AI) with timestamps.
  \item Supports retrieving previous sessions and exporting conversation history (introduced in Snapshot 2).
\end{itemize}

\subsubsection{AI Adapter Module}
\begin{itemize}
  \item Provides a single interface \texttt{sendMessageToAI(prompt, context)}.
  \item Abstracts away underlying AI provider (e.g., OpenAI, Anthropic).
  \item Allows switching providers without changing core business logic.
\end{itemize}

\subsubsection{Admin Module}
\begin{itemize}
  \item Provides admin-only endpoints to view usage statistics.
  \item Helps simulate monitoring and logging features.
\end{itemize}

\subsection{Data Model (Conceptual)}
Core entities:
\begin{itemize}
  \item \textbf{User}: id, name, email, hashed password, role (student/admin).
  \item \textbf{ChatSession}: id, user\_id, title, created\_at.
  \item \textbf{Message}: id, session\_id, sender (user/ai), content, created\_at.
  \item \textbf{Configuration}: key, value (for adjustable app settings).
\end{itemize}

These tables are not actually implemented in code, but they are described here to show how the system would be structured in a real implementation.

%==================================================
\section{User Interface Design}

\subsection{Main User Screens}
The main UI screens include:
\begin{itemize}
  \item \textbf{Login / Register Page}: Basic email+password form.
  \item \textbf{Chat Dashboard}: List of previous chat sessions, button to start a new chat.
  \item \textbf{Chat Screen}: Chat window with messages, input box, send button, and options to rename the session or export conversation.
  \item \textbf{Settings Page}: Toggles for theme (dark/light), language, and data retention preferences.
  \item \textbf{Admin Dashboard}: High-level view of total users, total messages, and average session length.
\end{itemize}

\subsection{UX Considerations}
\begin{itemize}
  \item Responsive design (works on desktop and mobile).
  \item Clear distinction between user messages and AI responses.
  \item Accessible fonts and contrast ratios.
  \item Loading indicators while the AI is responding.
\end{itemize}

%==================================================
\section{Workflow and Integration with Tools}

\subsection{Workflow Overview}
The high-level workflow is:
\begin{enumerate}
  \item User visits the ChatBox AI web UI and signs in.
  \item User creates or opens a chat session.
  \item User message is sent to the backend API.
  \item Backend builds context, calls the AI Adapter.
  \item AI response is returned, stored in the database, and displayed in the UI.
  \item Logs and metrics are recorded for future analysis.
\end{enumerate}

This workflow is visualized in the separate \texttt{workflow\_diagram.tex} document, which includes a figure or diagram (e.g., PNG) stored under \texttt{images/workflow.png}.

\subsection{Docker Integration}
\begin{itemize}
  \item The system is described in a \texttt{docker-compose.yml} file.
  \item Services: \texttt{frontend}, \texttt{backend}, \texttt{db}, and optional \texttt{proxy}.
  \item Environment variables define database connection strings and API keys (for a real implementation).
\end{itemize}

\subsection{Jira and TestRail}
\begin{itemize}
  \item Jira is used to plan sprints and tasks for each snapshot.
  \item TestRail is used to document test cases and test runs for Snapshots 2, 3, and 4.
  \item For this course project, we create test reports and export them as documents stored in this GitHub repository.
\end{itemize}

%==================================================
\section{Glossary}

\begin{description}
  \item[AI] Artificial Intelligence.
  \item[LLM] Large Language Model.
  \item[UI/UX] User Interface / User Experience.
  \item[API] Application Programming Interface.
  \item[DB] Database.
\end{description}

%==================================================
\section{References}

\begin{itemize}
  \item Course handouts and project description.
  \item Official documentation of Docker and docker-compose.
  \item Example articles and tutorials on AI chat applications.
\end{itemize}

\end{document}
